\documentclass[a4paper,10pt]{article}
\usepackage[utf8]{inputenc}
\usepackage[T1]{fontenc} % Use 8-bit encoding that has 256 glyphs
\usepackage{ae, aecompl}
\usepackage{amsmath,amsfonts,amsthm} % Math packages
\usepackage{enumitem}

\usepackage{fancyhdr} % Custom headers and footers

%opening
\title{14.2 Outer measures}

\begin{document}

\maketitle

\section*{14-11}
$\diamond$ \textbf{La mesure de Lebesgue sur $\mathbb{R}^d$ est une outer measure}
\\
\\
On suppose $d > 1$ car la chose est déjà prouvée pour $d = 1$. Soit alors $A \subseteq B$. Alors il est clair que tous les recouvrement de $B$ sont des recoubrement de $A$. Par le même raisonnement que dans la preuve de \textbf{thm. 8.6}, on peut déduire $\lambda (A) \leq \lambda (B)$.
\\
\\
On a de même que $\emptyset \subseteq (-\frac{1}{n}, \frac{1}{n})^d$. Alors $\lambda (\emptyset) \leq \lambda ([-\frac{1}{n}, \frac{1}{n}]^d)\leq (\frac{2}{n})^d$, qui est arbitrairement petit. Donc $\lambda(\emptyset) = 0$.
\\
\\
Pour la dernière partie de la preuve, on peut recopier mot à mot la démontration du \textbf{thm. 8.6} en imaginant que les $I_j^n$ sont des boites ouvertes.

\section*{14-12}
\begin{enumerate}
	\item $M$ un ensembles
	\item $\mu : \mathcal{P} (M) \rightarrow [0, \infty]$ une outer measure
\end{enumerate}
$\diamond$ \textbf{Pour tout $S,T \subseteq M$, on a $\mu (T) \leq \mu(T \cap S) + \mu (T \cap S^\prime)$}
\\
\\
Car on a que $T \subseteq (T \cap S) \cup (T\cap S^\prime)$. Par \textbf{def. 14.16}, on a $\mu (T) \leq \mu((T\cap S ) \cup (T \cap S^\prime)) \leq \mu(T \cap S) + \mu (T \cap S^\prime)$ (la subadditivité du cas où la séquence est fini est triviale).

\section*{14-13}
\begin{enumerate}
	\item $M$ un ensemble
	\item $\mu : \mathcal{P} (M) \rightarrow [0, \infty]$ une outer measure
\end{enumerate}
$\diamond$ \textbf{$\mu (S) = 0$ alors $S$ est $\mu$-mesurable}
\\
\\
Soit $T \subseteq M$. On a $\mu (T) \leq \mu(T \cap S ) + \mu (T \cap S^\prime) = \mu (T \cap S^\prime)$ par \textbf{prop. 14.21, def. 14.16}.
\\
\\
Or, $T \cap S^\prime \subseteq T$. Donc $\mu (T \cap S^\prime) \leq \mu (T)$ par \textbf{def. 14.16}. Donc $\mu(T) = \mu(T \cap S^\prime) = \mu (T \cap S^\prime) + \mu (T \cap S)$.

\section*{14-14}
\begin{enumerate}
	\item $M$ un ensemble
	\item $\mu : \mathcal{P} (M) \rightarrow [0, \infty]$ une outer measure 
	\item $A,B \subseteq M$
\end{enumerate}
$\diamond$ \textbf{Si $A,B$ sont $\mu$-mesurables, alors $A\cap B$ l'est aussi.}
\\
\\
Même preuve que \textbf{lem. 9.8}.

\section*{14-15}
Une copie de la preuve de \textbf{lem. 9.9}.

\section*{14-16}
\begin{enumerate}
	\item $M$ un ensemble
	\item $\mu \mathcal{P} (M) \rightarrow [0, \infty]$ une outer measure
	\item $\Sigma_\mu$ l'ensemble des ensembles $\mu$-mesurables
\end{enumerate}
$\diamond$ \textbf{$(M, \Sigma_\mu, \mu)$ est un espace mesuré}
\\
\\
Par \textbf{prop. 14.22}, $\emptyset \in \Sigma_\mu$. 
\\
\\
Soit alors $S \in \Sigma_\mu$ et $T \subseteq M$. On a $\mu (T) = \mu (T \cap S) + \mu (T \cap S^\prime) = \mu (T \cap (S^\prime)^\prime) + \mu (T \cap S^\prime)$ et donc $S^\prime$ est $\mu$-mesurable ie. $S^\prime \in \Sigma_\mu$.
\\
\\
De plus, par \textbf{lem. 14.24}, on a que $\bigcup_{n=1}^\infty A_n \in \Sigma_\mu$ si $A_n \in \Sigma_\mu$.
\\
\\
Donc $\Sigma_\mu$ est une $\sigma$-algèbre. On a que $\mu$ est une outer measure sur certains sous-ensembles de $M$, mais on veut une mesure sur $\Sigma_\mu$. On doit avoir l'additivité pour des ensembles disjoints.
\\
\\
Soit alors $\{A_n\}_{n=1}^\infty \subseteq \Sigma_\mu$ des ensembles disjoints. Par le \textbf{lem. 14.24}, on a 
\begin{align*}
	& \mu \left(\bigcup_{n=1}^\infty A_n \right) = \sum_{n=1}^\infty \mu \left(A_n \cap \left(\bigcup_{i=1}^\infty A_i \right)\right) + \mu \left(\left(\bigcup_{n=1}^\infty A_n \right)^\prime \cap \left(\bigcup_{n=1}^\infty A_n \right)\right) \\
	= \\
	& \sum_{n=1}^\infty \mu \left( A_n \cap \left(\bigcup_{i=1}^\infty A_i \right) \right) \\
	= & \quad <\textbf{pour tout } n, A_n \subseteq \bigcup_{i=1}^\infty A_i> \\
	& \sum_{n=1}^\infty \mu (A_n)	
\end{align*}
et donc on a l'additivité.

\section*{14-17}
\begin{enumerate}
	\item $d \leq 1$ et pour $i = 1, \cdots ,d $ on pose $a_i < b_i$ des nombres rationnelles dyadiques
	\item $D := \prod_{i=1}^d (a_i, b_i)$ une boite dyadique ouverte de $\mathbb{R}^d$
\end{enumerate}
$\diamond$ \textbf{Pour tout $S \subseteq \mathbb{R}^d$ on a }
\begin{align*}
	\lambda (S) = \inf \left\{ \sum_{n=1}^\infty |D_n| : S \subseteq \bigcup_{n=1}^\infty D_n, \text{ où chaque } D_n \text{ est une boite dyadique ouverte de } \mathbb{R}^d \right\}	
\end{align*}
\\
\\
On montre par induction que, pour toutes boites ouvertes $B$ il existe une boite ouverte dyadique $D$ telle que $|D| - |B| < \delta$ pour $\delta > 0$.
\\
\\
Si $d = 1$, la chose est évidente car, par \textbf{ex. 1-29c}, les nombres dyadiques sont denses dans $\mathbb{R}$.
\\
\\
On suppose la chose prouvée pour $d$ quelconque. On pose $\delta > 0$ quelconque. On considère
\begin{align*}
	& \prod_{n=1}^{d+1} (b^*_n - a^*_n) - \prod_{n=1}^{d+1} (b_n - a_n) \\
 	= \\
 	& (b^*_{d+1} - a^*_{d+1})\prod_{n=1}^d (b^*_n - a^*_n) - \prod_{n=1}^{d+1} (b_n - a_n)
\end{align*}
On pose $(b^*_{d+1} - a^*_{d+1}) - (b_{d+1} - a_{d+1}) \leq \frac{\delta}{2 \prod_{n=1}^d (b_n - a_n)}$ et on pose de plus $\delta^* := \frac{\delta}{2(b^*_{d+1} - a^*_{d+1})}$. (\textbf{hypothèse d'induction}).
\\
\\
On peut de plus, toujours par \textbf{hypothèse d'induction}, poser $\prod_{n=1}^d (b^*_n - a^*_n) \leq \delta^* + \prod_{n=1}^d (b_n - a_n)$. On a donc
\begin{align*}
	& (b^*_{d+1} - a^*_{d+1})\prod_{n=1}^d (b^*_n - a^*_n) - \prod_{n=1}^{d+1} (b_n - a_n) \\
	\leq \\
	& (b^*_{d+1} - a^*_{d+1})\delta^* + (b^*_{d+1} - a^*_{d+1})\prod_{n=1}^d (b_n - a_n) - (b_{d+1} - a_{d+1})\prod_{n=1}^d (b_n - a_n) \\
	= \\
	& (b^*_{d+1} - a^*_{d+1})\delta^* + \left( \prod_{n=1}^d (b_n - a_n)\right) ((b^*_{d+1} - a^*_{d+1}) - (b_{d+1} - a_{d+1})) \\
	\leq \\
	& \frac{\delta}{2} + \frac{\delta}{2} = \delta
\end{align*}
Ainsi donc, pour tout $d \in \mathbb{N}$ et pour toutes boites ouvertes $B$ il existe une boite ouverte dyadique $D$ tel que $|D| - |B| < \delta$ où $\delta > 0$.
\\
\\
On a que tout recouvrement de boites dyadiques ouvertes est un recouvrement et donc l'infimum des recouvrements dyadiques est $\geq$ à l'infimum des recouvrement de boites ouvertes. 
\\
\\
Supposons alors que l'infimum est $>$. Alors il exste $\epsilon$ la distance entre les deux. Soit alors $\{B_n\}_{n=1}^\infty$ un recouvrement de boites ouvertes tel que $\lambda (S) - \sum_{n=1}^\infty |B_n| < \frac{\epsilon}{2}$.
\\
\\
On veut construire un recouvrement de boites dyadique $D_n$ tel que $\sum_{n=1}^\infty |D_n| - \sum_{n=1}^\infty |B_n| \leq \frac{\epsilon}{2}$, car alors $|\lambda (S) - \sum_{n=1}^\infty |D_n| |  < \epsilon = \lambda^* (S) - \lambda (S)$ où $\lambda^* (S)$ est l'infimum des recouvrements dyadiques. On aura alors $\sum_{n=1}^\infty |D_n| < \lambda^* (S)$, une contradiction.
\\
\\
Or, pour tout $B_n$, il existe $D_n$ tel que $|D_n| - |B_n| \leq \frac{\epsilon}{2^{n+1}}$. Alors on a $\sum_{n=1}^\infty |D_n| - |B_n| \leq \sum_{n=1}^\infty \frac{\epsilon}{2^{n+1}} = \frac{\epsilon}{2}\sum_{n=1}^\infty \frac{1}{2^n} = \frac{\epsilon}{2}$.

\section*{14-18}
\begin{enumerate}
	\item $A,B \subseteq \mathbb{R}$
	\item $\lambda (A) = 0$
\end{enumerate}
$\diamond$ \textbf{$\lambda( A \times B) = 0$ où $\lambda$ est la mesure de Lebesgue sur $\mathbb{R}^2$}
\\
\\
On suppose d'abord que $\lambda (B) < \infty$. On pose $\{J_n\}_{n=1}^\infty$ tel que $B \subseteq \bigcup_{n=1}^\infty J_n$ et $\sum_{n=1}^\infty |J_n| - \lambda (B) < \epsilon^*$ pour un $\epsilon^* > 0$.
\\
\\
Soit alors $\epsilon > 0$. On pose $\{I_n\}_{n=1}^\infty$ tel que $A \subseteq \bigcup_{n=1}^\infty$ et $\sum_{n=1}^\infty |I_n| < \frac{\epsilon}{\lambda(B) + \epsilon^*}$.
\\
\\
Alors on a que $A \times B \subseteq \bigcup_{n=1}^\infty I_n \times \bigcup_{n=1}^\infty J_n = \bigcup_{n=1}^\infty \bigcup_{j=1}^\infty I_n \times J_j = \bigcup_{n,j = 1}^\infty I_n \times J_j$ et $\sum_{n,j=1}^\infty |I_n| |J_j| \leq \frac{\epsilon}{\lambda (B) + \epsilon^*} \sum_{j=1}^\infty |J_j| \leq \epsilon $.
\\
\\
Alors l'infimum est arbitrairement proche de 0 et donc $\lambda (A \times B) = 0$.
\\
\\
On a donc que, pour tout $n \in \mathbb{N}$, on a $\lambda (A \times [-n, n]) = 0$. Alors $\lambda (\bigcup_{n=1}^\infty A \times [-n,n]) \leq \sum_{n=1}^\infty \lambda (A \times [-n,n]) = 0$. Or $\bigcup_{n=1}^\infty A \times [-n,n] = A \times \mathbb{R}$.
\\
\\
Or $A \times B \subseteq A \times \mathbb{R}$. Donc $\lambda (A \times B) = 0$ pour tout $B \subseteq \mathbb{R}$.

\section*{14-19}
\begin{enumerate}
	\item $[a,b]$ un interval
	\item $S \subseteq [a,b]$ on définit $J(S) := \inf \{\sum_{j=1}^n |I_j| : S \subseteq \bigcup_{j=1}^n I_j \text{ où } I_j \textbf{ des intervals ouverts} \}$ le Jordan content.
\end{enumerate}
\subsection*{a)}
$\diamond$ \textbf{$[c,d] \subseteq [a,b]$ on a $J([c,d]) = d-c $}
\\
\\
Soit $\epsilon$. Alors il existe $\{I_n\}_{n=1}^\infty$ tel que $[c,d] \subseteq \bigcup_{n=1}^\infty I_n$ et $\sum_{n=1}^\infty |I_n| - \lambda([c,d]) < \epsilon$. Par \textbf{Heine-Borel}, il existe $\{I_{n_j}\}_{j=1}^m$ une sous suite telle que $[c,d] \subseteq \bigcup_{j=1}^m I_{n_j} \subseteq \bigcup_{n=1}^\infty I_n$.
\\
\\
Alors $\lambda([c,d]) \leq \sum_{j=1}^m |I_{n_j}| \leq \sum_{n=1}^\infty |I_n|$ et donc $\sum_{j=1}^m |I_{n_j}| - \lambda ([c,d]) \leq \epsilon$. 

\subsection*{b)}
$\diamond$ \textbf{Le Jordan content n'est pas une outer measure}
\\
\\
On pose $[a,b] = [0,1]$. Par \textbf{ex. 9-7}, on a $J(\mathbb{Q} \cap [0,1]) = J(\mathbb{Q}^\prime \cap [0,1]) = 1$. 
\\
\\
Supposons que $J$ était une outer measure. Alors $([0,1], \mathcal{P}([0,1]), J)$ serait un espace mesuré par \textbf{def. 14.7, exemple 14.2}.
\\
\\
Donc, par \textbf{prop. 14.11}, on a que $2 = J(\mathbb{Q} \cap [0,1]) + J (\mathbb{Q}^\prime \cap [0,1]) = J(\mathbb{Q} \cap [0,1] \cup \mathbb{Q}^\prime \cap [0,1]) = J ([0,1]) = 1$, une contradiction. 
\\
\\
Donc $J$ n'est pas une outer measure.

\subsection*{c)}
\begin{enumerate}
	\item $J_i (S) := \sup \{\sum_{j=1}^n |b_j - a_j| : S \supseteq \bigcup_{j=1}^n [a_j,b_j], a_1 \leq b_1 \leq a_2 \leq b_2 \leq \cdots \leq a_n \leq b_n\}$ le inner Jordan content de $S \subseteq [a,b]$.
\end{enumerate}
$\diamond$ \textbf{Pour $[c,d] \subseteq [a,b]$, on a $J_i ([c,d]) = d - c$}
\\
\\
Clairement, $|d - c| \leq J_i ([c,d])$. 
\\
\\
Aussi, je dis $b_i = a_{i+1}$.
\\
\\
Car sinon, on a $c = a_1 \leq b_1 \leq \cdots \leq a_i \leq b_i < a_{i+1} \leq \cdots \leq b_n = d$. Alors il existe $x \in [c,d]$ tel que $x \not \in \bigcup_{j=1}^n [a_j, b_j]$ si $x \in (b_i, a_{i+1})$.
\\
\\
Donc $b_i = a_{i+1}$.
\\
\\
Mais alors $\sum_{j=1}^n |b_j - a_j| = (b_1 - a_1) + (b_2 - a_2) + \cdots (b_n - a_n) = (a_2 - a_1) + (a_3 - a_2) + \cdots (b_n - a_n) = b_n - a_1 = d - c$. Par conséquent, toutes les partitions de cette forme donne $d - c$.

\subsection*{d)}
$\diamond$ \textbf{Le inner Jordan content n'est pas une outer measure}
\\
\\
Car $J_i (\emptyset)$ n'est pas définie. Car supposons $J_i (\emptyset) = x \in \mathbb{R}$. Alors il existe $a_1 \leq b_1 \leq \cdots \leq a_n \leq b_n$ tel que $\bigcup_{i=1}^n [a_i,b_i] \subseteq \emptyset$ et $|J_i (\emptyset) - \sum_{i=1}^n |b_i - a_i| | < \epsilon$.
\\
\\
Or, $\bigcup_{i=1}^n [a_i, b_i] \subseteq \emptyset$ est une contradiction.

\subsection*{e)}
\begin{enumerate}
	\item Soit une \textbf{algèbre} un ensemble d'ensembles tel que les deux premières propriétés d'une $\sigma$-algèbre sont satisfaites mais où seule la fermeture sous l'union finie d'ensembles s'applique plutôt que l'union dénombrable
	\item $S \subseteq \mathbb{R}$ jordan mesurable ssi $J(S) = J_i (S)$
\end{enumerate}
$\diamond$ \textbf{$\mathcal{J}_{[a,b]}$ l'ensemble des sous-ensembles jordan mesurables de $[a,b]$ est une algèbre}
\\
\\
Et immédiatement j'ai un problème car j'ai dis plus haut que $J_i (\emptyset)$ ne pouvait pas être 0 alors que $J (\emptyset) = 0$. Donc l'ensemble vide ne serait pas dans $\mathcal{J}_{[a,b]}$ et donc il ne s'agirait pas d'une algèbre.
\\
\\
Soit alors $S \in \mathcal{J}_{[a,b]}$. Alors $J(S) = J_i (S)$.
\end{document}


