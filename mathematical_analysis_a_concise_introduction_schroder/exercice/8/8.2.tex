\documentclass[a4paper,10pt]{article}
\usepackage[utf8]{inputenc}
\usepackage[T1]{fontenc} % Use 8-bit encoding that has 256 glyphs
\usepackage{ae, aecompl}
\usepackage{amsmath,amsfonts,amsthm} % Math packages
\usepackage{enumitem}

\usepackage{fancyhdr} % Custom headers and footers

%opening
\title{8.2 Lebesgue's Criterion for Riemann Integrability}

\begin{document}

\maketitle

\begin{abstract}
Exercices de la secion 8.2 sur la mesure de Lebesgue
\end{abstract}

\section*{8-7}
\begin{enumerate}
  \item $f : [a,b] \rightarrow \mathbb{R}$ bornée
  \item $I$, $J$, des intervals tel que $I \subseteq J$
\end{enumerate}
$\diamond$  $\omega_f (I) \leq \omega_f (J)$
\\
\\
On a que
\begin{align*}
  & \omega_f (I) = sup \{|f(z) - f(y)| : z,y \in I\} \\
  & \omega_f (J) = sup \{|f(z) - f(y)| : z,y \in J\}
\end{align*}
Soit $z,y \in I$. Alors $z,y \in J$. Alors $|f(z) - f(y)| \leq \omega_f (J)$, par définition du supremum.
\\
\\
Mais alors $\forall |f(z) - f(y)| \in \{|f(z) - f(y)| : z,y \in I\}$, $|f(z) - f(y)| \leq \omega_f (J)$. Alors 
$sup \{|f(z) - f(y)| : z,y \in I\} = \omega_f (I) \leq \omega_f (J)$ par définition du supremum.

\section*{8-8}
\subsection*{a)}
\begin{enumerate}
 \item $f(x) := \textbf{1}_{[0, \infty]}$
\end{enumerate}
$\diamond$ $\omega_f (0) = 1$
\\
\\
Soit $0 \in J$ un interval ouvert. Alors $inf(J) < 0$ et donc $\exists x \in J$ tel que $x < 0$. Mais alors 
$\textbf{1}_{[0, \infty]} (x) = 0$ et $\textbf{1}_{[0, \infty]} (x) = 1$. Donc $\omega_f (J) = 1$. 
\\
\\
Donc $\forall J$ ouvert tel que $0 \in J$, $\omega_f (J) = 1$. Donc $inf \{\omega_f (J) : 0 \in J \text{ ouvert} \} = 1 = \omega_f (0)$.

\subsection*{b)}
\begin{enumerate}
 \item $f : [a,b] \rightarrow \mathbb{R}$ bornée
 \item $x \in (a,b)$ tel que $\lim_{z\to x^+} f(z)$ et $\lim_{z \to x^-} f(z)$ existent
\end{enumerate}
$\diamond$ $\omega_f (x) \geq \left | \lim_{z \to x^-} f(z) - \lim_{z \to x^+} f(z) \right |$
\\
\\
On a que $\omega_f (x) \geq 0$. Donc, si $\left | \lim_{z \to x^-} f(z) - \lim_{z \to x^+} f(z) \right | = 0$, la proposition est prouvée.
\\
\\
Soit alors $\left | \lim_{z \to x^-} f(z) - \lim_{z \to x^+} f(z) \right | = L > 0$. (Implicitement, il s'agit d'un cas ou nous 
sommes dans un discontinuité en saut.)
\\
\\
Soit $J$ un ouvert tel que $x \in J$. Soit $\epsilon$, $L^+ = \lim_{z \to x^+} f(z)$ et $L^-  = \lim_{z \to x^-} f(z)$.
\\
\\
Alors il existe $\delta$ tel que $|y - x| < \delta$, $|z - x| < \delta$, $(x - \delta, x + \delta) \subseteq J$, 
$|f(z) - L^+| < \dfrac{\epsilon}{2}$ et $|f(y) - L^-| < \dfrac{\epsilon}{2}$.
\\
\\
Or, puisque $|L^+ - L^-| = L$, on a 
\begin{align*}
  & |L^+ - L^-| = |L^+ - f(z) - L^- + f(y) + f(z) - f(y)| \\
  \leq \\
  & |L^+ - f(z)| + |L^- - f(y)| + |f(z) - f(y)| \\
  = \\
  & \epsilon + |f(z) - f(y)|
\end{align*}
et donc $L - \epsilon \leq |f(z) - f(y)|$.
\\
\\
Donc $\forall J$ ouvert tel que $x \in J$, $\forall \epsilon$, il existe $z,y \in J$ tel que $|f(z) - f(y)| \geq L - \epsilon$.
\\
\\
Donc $\forall J$ ouvert tel que $x \in J$, $\sup \{|f(z) - f(y)| : z, y \in J \} \geq L$.
\\
\\
Donc $\omega_f (x) \geq L$.

\subsection*{c)}
$\textbf{Pour quel(s) cas l'inégalité en b) est-elle une égalité?}$
\\
\\
Dans une discontinuité en point, on a une inégalité stricte. Car alors la différence des limites est nulle, 
mais il est toujours possible de considérer $|f(x) - f(y)|$ où $y \not = x$. (Dans le mesure où $f(x)$ est définie.)
\\
\\
Dans une discontinuité en saut, on a l'égalité, car alors on peut choisir un ouvert suffisament petit autour de
$x$ pour borner toute différence $|f(z) - f(y)|$. 
\\
\\
Car soit $\epsilon$. Alors il existe $\delta$ tel que $z \in (x - \delta, x)$,
$y \in (x, x + \delta) \Rightarrow |f(z) - f(x^-)| < \dfrac{\epsilon}{2}$ et $|f(y) - f(x^+)| < \dfrac{\epsilon}{2}$.
\\
\\
Alors
\begin{align*}
 & |f(z) - f(x^-)| + |f(y) - f(x^+)| + |f(x^-) + f(x^+)| \leq L + \epsilon \\
 \leq \\
 & |f(z) - f(x^-) + f(y) - f(x^+) + f(x^-) + f(x^+)| \\
 = \\
 & |f(z) - f(y)| \leq L + \epsilon
\end{align*}
Donc pour tout $\epsilon$ il existe un ouvert $J$ contenant $x$ tel que pour tout $y,z \in J$, $|f(z) - f(y)| \leq L + \epsilon$. 
\\
\\
Alors $\omega_f (J) \leq L + \epsilon$. Donc, par le raisonnement fait en b), on a $L \leq \omega_f (J) \leq L + \epsilon$. 
\\
\\
Alors $\omega_f (x) = L$, qui est la différence des limites.
\\
\\
Ensuite, si la fonction est continue, la chose est évidente. Aussi, dans les deux autres types de discontinuité, l'une des deux limites
n'existe pas, et l'on est donc en dehors des hypothèses de la question b).

\section*{8-9}
\begin{enumerate}
 \item $f : [a,b] \rightarrow \mathbb{R}$ tel que $x \in [a,b]$
\end{enumerate}
$\diamond$ $\omega_f (x) = \lim_{n \to \infty} \omega_f \left(x - \dfrac{1}{n}, x + \dfrac{1}{n} \right)$
\\
\\
On pose $\omega_f \left (x - \dfrac{1}{n}, x + \dfrac{1}{n} \right) =: \omega_f (I_n)$. 
Alors, de ce que $I_n$ est un ouvert contenant $x$, on a que $\forall n \in \mathbb{N}$, $\omega_f (I_n) \geq \omega_f (x)$ 
par définition de l'infimum et donc $\lim_{n \to \infty} \omega_f (I_n) \geq \omega_f (x)$. 
\\
\\
Supposons alors que $\lim_{n \to \infty} \omega_f (I_n) > \omega_f (x)$. Alors 
\begin{align*}
  & \exists \epsilon \text{ tel que } \forall n \in \mathbb{N} \text{    } \omega_f (I_n) - \omega_f (x) > \epsilon
\end{align*}
Or, il existe $J$ tel que $\omega_f (J) - \omega_f (x) < \epsilon$. Or, il existe $n \in \mathbb{N}$ tel que 
$I_n \subseteq J \Rightarrow \omega_f (I_n) \leq \omega_f (J)$ par l'exercice $\textbf{8-7}$. Mais alors 
$\omega_f (I_n) - \omega_f (x) \leq \omega_f (J) - \omega_f (x) \leq \epsilon$, une contradiction.
\\
\\
Donc pour tout $\epsilon$ il existe $n \in \mathbb{N}$ tel que $\omega_f (I_n) - \omega_f (x) < \epsilon$, ie. 
$\lim_{n \to \infty} \omega_f (I_n) = \omega_f (x)$.

\section*{8-10}
\begin{enumerate}
 \item $f(x) := \begin{cases}
 0 & \text{ for } x \geq 0 \\
 1 & \text{ for } x < 0
\end{cases}$
\end{enumerate}
$\diamond$ $\inf \{ \omega_f (J) : x \in J \text{ un interval }\} = 0$
\\
\\
Car $J$ peut être fermé. On pose $J := [0,0]$. 

\section*{8-11}
\begin{enumerate}
 \item $F : [a,b] \rightarrow \mathbb{R}$ une fonction non décroissante
\end{enumerate}
$\diamond$ $F \textbf{ est riemann}$
\\
\\
Car puisque $F$ est non décroissante, alors par $\textbf{7-23}$ elle possède au plus un nombre 
dénombrable de discontinuités. Alors $F$ est continue pp, puisque les points auxquels elle ne peut 
l'être forme un ensemble de mesure nulle ($\textbf{prop 8.3}$).
\\
\\
Or, les fonctions continues pp sont riemann ($\textbf{thm. 8.12}$).

\section*{8-12}
\begin{enumerate}
 \item $f:[a,b] \rightarrow \mathbb{R}$ 
 \item $a < b$
 \item $V^b_a f := \sup \left\{ \sum\limits_{i=1}^n |f(a_i) - f(a_{i-1})| : a=a_0 < a_1 \dots < a_n = b \right\} < \infty$ c-a-d dont la variation est bornée.
\end{enumerate}

\subsection*{a)}
$\diamond$ $\forall f \textbf{ de variation bornée } \exists h, g \textbf{ non décroissantes tel que } f = g-h$
\\
\\
Si l'on montre que $V_a^x f$ est non décroissante, puis que $V_a^x f - f(x)$ est non décroissante, on pourra conclure
que $f(x) = V_a^x f - (V_a^x - f(x))$ est la différence de deux fonctions non décroissantes.
\\
\\
$\textbf{Que } V_a^x f \textbf{ est non décroissante}$
\\
\\
Soit $x < y$. Alors $|f(y) - f(x)|$ est fixé et alors on pose $0 \leq \epsilon \leq |f(y) - f(x)|$.
\\
\\
Soit $P$ une partition de $[a,x]$ tel que $V^x_a f - \sum\limits_{i=1}^n |f(a_i) - f(a_{i-1})| \leq \epsilon$ ($\textbf{thm. 1.21}$). 
\\
\\
On pose $P'$ tel que $a = a_0 < \dots < a_n  = x < a_{n+1} = y$.
\\
\\
Alors 
\begin{align*}
 & V_a^x f - \sum\limits_{i=1}^n |f(a_i) - f(a_{i-1})| \leq \epsilon  = |f(y) - f(x)| \\
 = \\
 & \sum\limits_{i=1}^{n+1}|f(a_i) - f(a_{i-1})| - \sum\limits_{i=1}^n |f(a_i) - f(a_{i-1})|
\end{align*}
Alors $V_a^x f \leq \sum\limits_{i=1}^{n+1}|f(a_i) - f(a_{i-1})| \leq V_a^y f$. Donc $V_a^x f$ est non décroissante.
\\
\\
$\textbf{Que } V_a^x f - f(x) \textbf{ est non décroissante}$
\\
\\
On montre d'abord que 
\begin{align*}
 \forall z \in (a,x) V_a^x f \geq V_a^z f + V_z^x f
\end{align*}
Car soit $a_0 = a < \dots < a_n = z$ et $a_n = z < \dots a_{n+k} = x$ une partition de $[a,z]$ et de $[z,x]$ respectivement.
\\
\\
Alors $a_0 = a < \dots < a_n+k = x$ est une partition de $[a,x]$. Donc $\sum\limits_{i=1}^n |f(a_i) - f(a_{i-1})| +
\sum\limits_{i = n}^{n+k} |f(a_i) - f(a_{i-1})| \leq V_a^x f$ par définition du supremum.
\\
\\
On a donc que $ \forall P$ partition de $[a,z]$, $P'$ partition de $[z,x]$
\begin{align*}
 \sum\limits_{P} |f(a_i) - f(a_{i-1})| + \sum\limits_{P'} |f(a_i) - f(a_{i-1})| \leq V_a^x f
\end{align*}
Soit alors $V_a^z f + V_z^x f - V_a^x f = \epsilon > 0$, ce que l'on peut supposer puisque chacune
de ces quantités est finies par 3. et ($0 \leq V_a^b f < \infty \textbf{ et } V_a^x f \textbf{ est non décroissante}$).
\\
\\
Alors il existe ($\textbf{thm. 1.21}$) $P, P'$ des partitions de $[a,z]$, $[z,x]$ respectivement telles que 
\begin{align*}
 & V_a^z f - \sum\limits_P |f(a_i) - f(a_{i-1})| < \frac{\epsilon}{2} \\
 & V_z^x f - \sum\limits_{P'} |f(a_i) - f(a_{i-1})| < \frac{\epsilon}{2}
\end{align*}
Alors 
\begin{align*}
 & V_a^z f + V_z^x f - V_a^x f < \epsilon + \sum\limits_P |f(a_i) - f(a_{i-1})| + \sum\limits_{P'} |f(a_i) - f(a_{i-1})| - V_a^x f \\
 \Rightarrow \\
 & 0 < \sum\limits_P |f(a_i) - f(a_{i-1})| + \sum\limits_{P'} |f(a_i) - f(a_{i-1})| - V_a^x f \\
 \Rightarrow \\
 & V_a^x f < \sum\limits_P |f(a_i) - f(a_{i-1})| + \sum\limits_{P'} |f(a_i) - f(a_{i-1})|
\end{align*}
une contradiction par rapport à ce qui fut montré plus haut.
\\
\\
Donc $\forall z \in (a,x) V_a^x f \geq V_a^z f + V_z^x f \Leftrightarrow V_a^x f - V_a^z f \geq  V_z^x f$, une expression
bien définie puisque chacun de ses termes est fini ($\textbf{par 3.}$).
\\
\\
Finalement, supposons $x < y$. Alors 
\begin{align*}
 & V_a^x - f(x) \leq V_a^y - f(y) \\
 \Leftrightarrow \\
 & f(y) - f(x) \leq V_a^y f - V_a^x f
\end{align*}
Or, on a que $x < y$ est une partition de $[x,y]$. Donc $f(y) - f(x) \leq |f(y) - f(x)| \leq V_x^y f \leq V_a^y f - V_a^x f$.
\\
\\
Donc $V_a^x f - f(x)$ est une fonction non décroissante.

\subsection*{b) - c)}
$\diamond$ $\textbf{Toute fonction } f \textbf{ de variation bornée est riemann}$
\\
\\
Soit $f$ une fonction de variation bornée. 
\\
\\
Alors il existe $g,h$ non décroissante tel que $f = g-h$ ($\textbf{par 8-12 a)}$).
C-a-d que $f = g+(-h)$. 
\\
\\
Or, $g$ est continue pp puisque non-décroissante ($\textbf{ex. 7-23}$). Aussi, $-h$ est continue presque partout puisque
$h$ est continue pp ($\textbf{ex. 7-23 + s.d.}$). $f$ est donc la somme de fonctions continues pp, donc elle est continue pp ($\textbf{ex. 8-6}$).
\\
\\
Alors $f$ est riemann ($\textbf{thm. 8.12}$).
\\
\\
NOTE : On aurait pû éviter le $\textbf{s.d.}$ en montrant que les discontinuité ne pouvait subsister que sur des ensembles de mesure nulles,
mais on aurait alors repris le raisonnement fait en 8-6.

\subsection*{d)}
\begin{enumerate}
 \item $f,g$ des fonctions à variation bornée
 \item $\alpha \in \mathbb{R}$
\end{enumerate}
$\diamond$ $f+g \textbf{ et } \alpha f \textbf{ sont à variation bornée.}$ 
\\
\\
Pour $f+g$, on considère la quantité $V_a^b f + V_a^b g < \infty$. Il est ensuite aisé de montré que,
pour toute partition, on peut combiner les sommes et, par inégalité triangulaire, obtenir que la quantité
résultante est inférieure à celle définie plus ci-haut.
\\
\\
Pour $\alpha f$, le résultat suit de ce que $\sup(\alpha A) = \alpha \sup(A)$.

\section*{8-13}
Une fonction $f : [a,b] \rightarrow \mathbb{R}$ est continue par morceau ssi existe une partition
$a = z_0 < z_1 < \dots < z_n = b$ tel que pour tout $i = 1 \dots n$, la restriction $\textbf{1}_{(z_{i-1}, z_i)}f$ est continue.
\begin{enumerate}
 \item $f : [a,b] \rightarrow \mathbb{R}$ continue par morceau
\end{enumerate}
$\diamond$ $f \textbf{ est riemann}$
\\
\\
Car soit $\{z_0 \dots z_n\}$ la partition de $[a,b]$ sur laquelle les restrictions de $f$ sont continues. 
\\
\\
Soit $x \in [a,b]$ tel que $f$ n'est pas continue. Alors $x \not\in \bigcup\limits_{i=1}^n (z_{i-1}, z_i)$ ($\textbf{[1]}$).
\\
\\
Alors $x \in \{z_0 \dots z_n\}$ un ensemble dénombrable. Donc l'ensemble des points pour lesquels $f$ est discontinue
est de mesure nulle ($\textbf{prop. 8.3}$). Donc $f$ est continue pp ($\textbf{thm. 8.12}$).

\section*{8-14}
\begin{enumerate}
 \item $A, N \subseteq \mathbb{R}$
 \item $\lambda (N) = 0$
\end{enumerate}
$\diamond$ $\lambda(A-N) = \lambda(A)$
\\
\\
Car 
\begin{align*}
  & A - N \subseteq A \\
  \Rightarrow \\
  & \lambda(A-N) \leq \lambda(A) = \lambda(A - N \cup N) \\
  \leq \\
  & \lambda(A - N) + \lambda(N) = \lambda(A - N)
\end{align*}

\section*{8-15}
\begin{enumerate}
 \item $C^Q$ un ensemble de cantor de mesure non nulle
\end{enumerate}
$\diamond$ $\textbf{1}_{C^Q} \textbf{ n'est pas riemann}$
\\
\\
On se limitera à considérer le Cantor sur $[0,1]$.
\\
\\
On montre d'abord que $0 < |I_{i,n}| < \dfrac{1}{2^n}$ pour tout $n$.
\\
\\
Car pour un premier interval $[0,1]$, on a que $C^Q_1 = [0, q] \cup [1 - q, 1]$.
\\
\\
Or, $|[0, q]| = q \in (0, \dfrac{1}{2})$ et de même pour l'interval de droite.
\\
\\
Supposons alors la chose prouvée pour $n$. On pose $I_{i,n} =: [a,b]$ puis on
considère $I_{2i-1, n+1} = L_{q_{n+1}} \left[ I_{i,n}\right] = [a, a + q_{n+1}(b-a)]$.
\\
\\
Or, la longueur de ce dernier interval est donnée par $q_{n+1}(b-a) < q_{n+1}\dfrac{1}{2^n} < \dfrac{1}{2}\dfrac{1}{2^n} = \dfrac{1}{2^{n+1}}$,
ce qui conclu la preuve.
\\
\\
Soit alors $x \in C^Q$ et $\delta > 0$. Il existe un $n$ tel que $\dfrac{1}{2^n} < \delta$. 
\\
\\
Or, puisque $x \in C^Q$, $x \in \bigcap\limits^\infty C^Q_m$. On pose $m := n+1$.
\\
\\
Alors il existe $i$ tel que $x \in I_{i,n+1}$. Or je dis que $I_{i,n+1} \subset (x - \delta, x + \delta)$. 
Car sinon soit $y \in I_{i,n+1} - (x - \delta, x + \delta)$. Alors $y - x > \delta$. 
Or $y - x < \dfrac{1}{2^{n+1}} < \dfrac{1}{2^n} < \delta$, une contradiction.
\\
\\
Je suppose maintenant, sans démonstration, car ce serait fastidieux, que de part et d'autre de $I_{i, n+1}$, il existe
$z \not\in C^Q$ tel que $z \in (x - \delta, x + \delta)$. Cela vient de ce qu'entre les divers $I_{i-1, n+1}$, $I_{i+1, n+1}$,
il existe des intervalles n'étant pas inclus dans le Cantor. 
\\
\\
Alors $|\textbf{1}_{C^Q}(x) - \textbf{1}_{C^Q}(z)| = 1$.
\\
\\
On a donc montré que $\forall x \in C^Q \forall \delta > 0 \exists z \in (x - \delta, x + \delta) 
\text{ tel que } |\textbf{1}_{C^Q}(x) - \textbf{1}_{C^Q}(z)| = 1$.
\\
\\
Cela montre que $\textbf{1}_{C^Q}(x)$ est discontinue sur $C^Q$. Si ce dernier Cantor est de mesure non nulle,
alors sa fonction indicatrice ne peut pas être Riemann.

\end{document}
