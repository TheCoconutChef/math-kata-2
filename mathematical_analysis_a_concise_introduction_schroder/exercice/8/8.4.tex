\documentclass[a4paper,10pt]{article}
\usepackage[utf8]{inputenc}
\usepackage[T1]{fontenc} % Use 8-bit encoding that has 256 glyphs
\usepackage{ae, aecompl}
\usepackage{amsmath,amsfonts,amsthm} % Math packages
\usepackage{enumitem}

\usepackage{fancyhdr} % Custom headers and footers

%opening
\title{8.4 Improper Riemann Integrals}

\begin{document}

\maketitle

\section*{8-33}
\begin{enumerate}
 \item $p > 0$ un rationnel
 \item $f(x) := \dfrac{1}{x^p}$
\end{enumerate}
$\diamond$ \textbf{$f$ est riemann impropre sur $(0,1]$ ssi $p < 1$}  
\\
\\
Si $p = 1$, alors $\int_0^1 f = \lim_{c \to 0}(-\log(c))$ qui tend vers $-\infty$.
\\
\\
Soit alors $p < 1$. On a
\begin{align*}
 & \lim_{t \to 0} \int_t^1 f \\
 = \\
 & \lim_{t \to 0}(\dfrac{1}{1-p} - \dfrac{t^{1-p}}{1-p}) \\
 = \\
 & \dfrac{1}{1-p}(1 - \lim_{t \to 0} t^{1-p}) 
\end{align*}
Or $1-p$ est positif est donc $\lim_{t \to 0} t^{1-p} = 0$ (\textbf{s.d.}).
\\
\\
Mais cette limite ne peut converger si $p \geq 1$ est donc si $f$ est riemann impropre, $p < 1$ (\textbf{s.d.}).

\section*{8-35}
\begin{enumerate}
 \item $f:[a,b) \rightarrow \mathbb{R}$ où $b \in \mathbb{R} \cup \{\infty\}$
 \item $f$ riemann pour tout $[a,c] \subseteq [a,b)$
\end{enumerate}
$\diamond$ \textbf{$f$ est riemann impropre ssi pour tout $c \in [a,b)$ $f$ est riemann impropre sur $[c,b)$ et alors 
$\int_a^b f = \int a^c f + \int c^b f$}
Soit $f$ riemann impropre sur $[c,b)$ pour tout $c \in [a,b)$. Alors
\begin{align*}
 & \int_a^c f + \int_c^b f \\
 = \\
 & \lim_{t \to c} \int_a^t f + \lim_{t \to c} \int_c^b f \\
 = \\
 & \lim_{t \to c}(\int_a^t f + \int_t^b f) \\
 = \\
 & \lim_{t \to c}(\int_a^b f) = \int_a^b f
\end{align*}
par des applications de \textbf{thm. 3.10} et \textbf{thm. 5.8}.
\\
\\
Soit alors $f$ riemann impropre sur $[a,b)$. Supposons de plus qu'il existe un $c$ tel que 
$f$ n'est pas riemann impropre sur $[c,b)$. 
\\
\\
Alors $h(z) := \int_a^c f$ est bornée puisque $f$ est riemann sur tout $[a,c] \subseteq [a,b)$. 
\\
\\
Soit alors $g(z) := \int_c^z f$. Alors (\textbf{spdg}) $\lim_{z \to b} g(z) = \infty$. On donc
\begin{align*}
 & \int_a^b f = \lim_{z \to b} \int_a^z f \\
 = \\
 & \lim_{z \to b} (\int_a^c f + \int_c^z f) \\
 = \\
 & \lim_{z \to b}(h(z) + g(z))
\end{align*}
Or, par \textbf{ex. 3-22}, cette limite est égale à $\infty$. Donc, $\int_a^b f = \infty$, une contradiction.
\\
\\
Donc pour tout $c \in [a,b)$, $f$ est riemann impropre sur $[c,b)$. 
\\
\\
L'argument présenté plus haut est alors applicable pour montrer que $$\int_a^c f + \int_c^b f = \int_a^b f$$.

\section*{8-37}
\begin{enumerate}
 \item $f:[a,b) \rightarrow \mathbb{R}$ où $b \in \mathbb{R} \cup \{\infty\}$
 \item $f$ riemann sur tout $[a,c] \subseteq [a,b)$
 \item $|f|$ riemann impropre sur $[a,b)$
\end{enumerate}
$\diamond$ \textbf{$f$ est riemann impropre sur $[a,b)$ et $\left| \int_a^b f \right| \leq \int_a^b |f|$}
\\
\\
On a 
\begin{align*}
 & 0 \leq f + |f| \leq 2|f| \\
 \Rightarrow \\
 & 0 \leq \int_a^c f + |f| \leq 2 \int_a^c |f| \\
 \Rightarrow \\
 & -\int_a^c |f| \leq \int_a^c f \leq \int_a^c |f| \\
 \Rightarrow \\
 & -\int_a^b |f| \leq \lim_{c \to b} \int_a^c f \leq \int_a^b |f|
\end{align*}
par \textbf{prop. 5.21 (monotonie de l'intégrale)}. Donc $\int_a^b f$ existe. 
\\
\\
Alors 
\begin{align*}
 & \left|\int_a^c f\right| \leq \int_a^c |f| \quad \text{pour tout $c \in [a,b)$} 
\end{align*}
\textbf{(thm. 8.14)} et donc 
\begin{align*}
 & \lim_{c \to b}\left|\int_a^c f \right| \leq \int_a^b |f|
\end{align*}
Or, puisque $\int_a^b f$ existe, on a que $\lim_{c \to b} \left |\int_a^c f\right| = \left| \lim_{c \to b} \int_a^c f \right |$ par 
\textbf{ex. 2-12}.

\section*{8-38}
$\diamond$ \textbf{Construire $f:[1,\infty) \rightarrow [0,1]$ tq $f$ est riemann impropre mais $\lim_{x \to \infty} f(x) \not = 0$}
\\
\\
On définit
\begin{align*}
 & I_i := [i, i+\frac{1}{2^{i+1}}) \text{ où $i \in \{1 \cdots \}$}\\
 & f_i := \left\{
    \begin{array}{l}
      (x - i)2^{i+1} \mbox{ si } x \in I_i \\
      0 \mbox{ sinon } \\ 
    \end{array}
    \right. \\
 & f := \sum_{i=1}^{\infty} f_i
\end{align*} 
$f$ définit une série de triangle de base $\frac{1}{2^{i+1}}$ et de hauteur 1, donc d'air $\frac{1}{2^i}$ (\textbf{s.d.}).
\\
\\
De plus, $f$ ne tend pas vers $0$ lorsque $x \rightarrow \infty$. Car $\lim_{x \to \infty} f(x) = 0$ ssi $\forall \{a_n\}$ tq
$a_i \rightarrow \infty$ $(i \rightarrow \infty)$, on a $\lim_{i \to \infty} f(a_i) = 0$ (\textbf{def. 3.1}).
\\
\\
Soit alors $a_i := i + \frac{1}{2^{i+1}}$. Alors on a $f(a_i) = (i + \frac{1}{2^{i+1}} - i)2^{i+1} = 1$ pour tout $i$ (\textbf{def.}). 
De plus, $a_i \rightarrow \infty$ $(i \rightarrow \infty)$, car $a_i > i$ (\textbf{def.}).
\\
\\
Soit alors $c \in [1, \infty)$. Alors il existe $i(c) \in \mathbb{N}$ minimum tel que $i(c) \geq c$ (\textbf{thm. 1.32, thm. 1.27}).
\\
\\
Alors $\lim_{c \to \infty} i(c) = \infty$ (\textbf{s.d.}). Puisque $f \geq 0$ (\textbf{def.}) on a que $F(t) := \int_1^t f$ est non
décroissante (\textbf{s.d.}). On a alors
\begin{align*}
  & \int_1^c f \leq \int_1^{i(c)} f \\
  = \\
  & \sum_{j=1}^{i(c)} \int_j^{j+\frac{1}{2^{j+1}}} \sum_{i=1}^\infty f_i \\
  = \\
  & \sum_{j=1}^{i(c)} \int_j^{j+\frac{1}{2^{j+1}}} f_j \\
  = \\
  & \sum_{j=1}^{i(c)} \frac{1}{2^j}
\end{align*}
Alors 
\begin{align*}
 & \lim_{c \to \infty} \int_1^c f \leq \lim_{c \to \infty}\sum_{j=1}^{i(c)} \frac{1}{2^{j+1}} \\
 = \\
 & \sum_{j=1}^{\infty} \frac{1}{2^j} = 1
\end{align*}
et donc l'intégrale impropre existe (\textbf{thm. 2.37 (thm. séquences monotones)}).

\section*{8-39 : Critère de Cauchy}
\begin{enumerate}
 \item $f:[a,b) \rightarrow [0,\infty)$ où $b \in \mathbb{R} \cup \{\infty\}$
 \item $f$ riemann pour tout $[a,c] \subseteq [a,b)$
\end{enumerate}
$\diamond$ \textbf{$f$ est riemann impropre sur $[a,b)$ ssi pour tout $\epsilon >0$ il existe $M \in [a,b)$ tel que pour tout 
$c,d \in (M,b)$ on a $\left | \int_c^d f \right| < \epsilon$}
\\
\\
On pose $F(t) := \int_a^t f$. On a que 
\begin{align*}
 \forall \epsilon \exists M \text{ tq } d(t,b) < M \Rightarrow \left|F(t) - \lim_{x \to b} \int_a^x f \right| < \frac{\epsilon}{2}
\end{align*}
Soit alors $t_1,t_2 \in (M, b)$ et, spdg, $t_1 < t_2$. Alors 
\begin{align*}
 & \left|\lim_{x \to b} \int_a^x f - F(t_1) \right | + \left | F(t_2) - \lim_{x \to b} \int_a^x f \right | < \epsilon \\
 \Rightarrow \\
 & \left | F(t_2) - F(t_1) \right | < \epsilon \\
 \Rightarrow \\
 & \left | \int_{t_1}^{t_2} f \right | < \epsilon 
\end{align*}

\section*{8-40}
\begin{enumerate}
 \item $f,g : [a,b) \rightarrow [0, \infty)$ où $b \in \mathbb{R} \cup \{\infty\}$
 \item $f,g$ riemann pour tout $[a,c] \subseteq [a,b)$
 \item $\lim_{x \to b^-} \dfrac{f(x)}{g(x)} = K > 0$
\end{enumerate}
$\diamond$ \textbf{$\int_a^b f$ conversge ssi $\int_a^b g$ converge}
\\
\\
\textit{Indice : } Près de $b$, on a que $g(x)(K - \epsilon) \leq f(x) \leq g(x)(K + \epsilon)$
\\
\\
On montre facilement l'énoncé donné par l'indice à partir de l'hyposthèse [3]. Supposons alors que 
$\int_a^b g$ existe et supposons de plus $M$ tq pour tout $x \in (M,b)$, on ait 
$g(x)(K - \epsilon) \leq f(x) \leq g(x)(K + \epsilon)$. Alors 
\begin{align*}
 & (K-\epsilon) \int_M^t g \leq \int_M^t f \leq (K + \epsilon) \int_M^t g \\
 \Rightarrow \\
 & (K - \epsilon) \int_M^b g \leq \int_M^b f \leq (K + \epsilon) \int_M^b g
\end{align*}
par \textbf{monotonicité de l'intégrale} et \textbf{monotonicité de la limite}. On a de plus, par [2], que $\int_a^M f$ est riemann.
Alors 
\begin{align*}
 & \int_a^M f + \int_M^b f \\
 = \\
 & \int_a^M f + \lim_{x \to b} \int_M^x f \\
 = \\
 & \lim_{x \to b} \int_a^M f + \lim_{x \to b} \int_M^x f \\
 = \\
 & \lim_{x \to b} \left ( \int_a^M f + \int_M^x f \right ) \\
 = \\
 & \lim_{x \to b} \int_a^x f = \int_a^b f
\end{align*}
Supposons alors que $\int_a^b f$ existe. Alors pour tout $M$, $\int_M^b f$ existe (\textbf{s.d.}). Supposons alors $M$ tel qu'il fut fait
ci-haut. Alors 
\begin{align*}
 & (K - \epsilon) \int_M^t g \leq \int_M^t f \\
 \Rightarrow \\
 & (K - \epsilon) \int_M^b g \leq \int_M^b f
\end{align*}
et donc $\int_M^b g$ existe. Puisque $\int_a^M g$ existe par [2], on peut conclure.

\section*{8-41}
\begin{enumerate}
 \item $f : (a,b] \rightarrow \mathbb{R}$
 \item $g := x^{-2} f(a + \frac{1}{x})$
\end{enumerate}
$\diamond$ \textbf{$f$ est riemann impropre sur $(a,b]$ ssi $g$ est riemann impropre sur $[\frac{1}{b-a}, \infty)$ 
et alors les intégrales sont égales}
\\
\\
Supposons alors $f$ riemann impropre sur $(a,b]$. Alors, posant $y := \frac{1}{x - a}$, on a 
\begin{align*}
 & \lim_{t \to a} \int_t^b f(x) dx \\
 = \\
 & \lim_{t \to a} \int_\frac{1}{t-a}^\frac{1}{b-a} f(a + \frac{1}{y})(-y^{-2}) dy 
\end{align*}
La fonction $-x^{-2}f(a + \frac{1}{x})$ est donc riemann impropre sur $[\frac{1}{b-a}, \infty)$ (\textbf{def.}) et donc
$g$ est riemann impropre sur ce même interval (\textbf{thm. 8.25}). 
\\
\\
La substitution inverse dans l'autre sens devrait nous asssurer que si $g$ est riemann impropre, alors $f$ l'est également.
\end{document}
