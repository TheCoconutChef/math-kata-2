\documentclass[a4paper,10pt]{article}
\usepackage[utf8]{inputenc}
\usepackage[T1]{fontenc} % Use 8-bit encoding that has 256 glyphs
\usepackage{ae, aecompl}
\usepackage{amsmath,amsfonts,amsthm} % Math packages
\usepackage{enumitem}

\usepackage{fancyhdr} % Custom headers and footers

%opening
\title{9.3 The Lebesgue Integral}

\begin{document}

\maketitle

\section*{9-20}
\subsection*{a)}
\begin{enumerate}
 \item $s$ une fonction simple non-négative
 \item $y_1, \cdots, y_m \in (0, \infty)$ les valeurs non-nulles prisent par $s$
 \item $a_1, \cdots, a_n \in [0, \infty)$ et $A_1, \cdots , A_n \subseteq \mathbb{R}$ des ensembles Lebesgue mesurables deux à deux disjoints tel que $s = \sum_{k=1}^n a_k \chi_{A_k}$
\end{enumerate}

$$
   \diamond	\sum_{k=1}^n a_k \lambda ( A_k ) = \sum_{j=1}^m y_j \lambda(s^{-1} (y_j))	
$$
Pour chaque $a_k$, soi $a_k = 0$, soi $a_k > 0$ et alors $a_k = y_i$ pour un certain $i \in \{1 , \cdots, m \}$. On désigne par $\{a_{i_1}, \cdots, a_{i_l}\}$ les $a_{i_k}$ tel que $a_{i_k} = y_i$. On définit de même $B_i := \bigcup_{j=1}^l A_{i_j}$. 
\\
\\
Puisque les $A_k$ sont deux à deux disjoints, on a $\lambda (B_i) = \sum_{j=1}^l A_{i_j}$. On peut donc réarranger 
\begin{align*}
	& \sum_{k=1}^n a_k \lambda (A_k) \\
	= \\
	& \sum_{j=1}^m \sum_{i=1}^l a_{j_i} \lambda (A_{j_i}) \\
	= & \qquad < a_{j_i} = y_j \textbf{ par def} >\\
	& \sum_{j=1}^m y_j \sum_{i=1}^l \lambda	(A_{j_i}) \\
	= \\
	& \sum_{j=1}^m y_j \lambda (B_j)
\end{align*}
On doit donc montrer que $B_j = s^{-1} (y_j)$. On a
\begin{align*}
	& x \in B_j \\
	\Leftrightarrow \\
	& x \in \bigcup_{i = 1}^l A_{j_i} \\
	\Leftrightarrow \\
	& s(x) = a_{j_i} = y_j
\end{align*}

\subsection*{b)}
\begin{enumerate}
	\item $s_1, s_2$ des fonctions simples
\end{enumerate}
$\diamond$ \textbf{$\int_{\mathbb{R}} (s_1 + s_2) d \lambda = \int_{\mathbb{R}} s_1 d \lambda + \int_{\mathbb{R}} s_2 d \lambda$}
\\
\\
Si $s_1, s_2$ sont Lebesgue intégrales, on applique \textbf{thm. 9.25}.
\\
\\
Soit $s_1, s_2 \geq 0$ tq $\int_\mathbb{R} s_1 d\lambda = \infty$. On doit montrer que $\int_\mathbb{R} (s_1 + s_2) d\lambda = \infty$.
\\
\\
Premièrement, on a qu'il existe $B_i$ tel que $\lambda (B_i) = \infty$ et $b_i \neq 0$.
\\
\\
On définit $A_j := (s_1 + s_2)^{-1}(y_j)$ pour des $y_j$ non nulle et on a donc $\int_\mathbb{R} (s_1 + s_2) d\lambda = \sum_{j=1}^n y_j \lambda (A_j)$ par \textbf{ex. 9-20a}.
\\
\\
Soit $A_{j_1} \cdots A_{j_k}$ les ensembles tels que $B_i \cap A_{j_l} \neq \emptyset$.
\\
\\
Alors je dis qu'il existe $m \in \{1 \cdots k\}$ tel que $\lambda (A_m) = \infty$.
\\
\\
Car sinon, ces ensembles sont tous de mesure finie. Or $B_i \subseteq \bigcup_{l=1}^k A_{j_l}$. 
\\
\\
Car soit $x \in B_i$. Alors $(s_1 + s_2)(x) = y_t$ pour un certain $t$. Mais alors $x \in A_t$ et donc $x \in \bigcup_{l=1}^k A_{j_l}$. 
\\
\\
Donc $B_i \subseteq \bigcup_{l=1}^k A_{j=l}$ et donc $\lambda (B_i) = \infty \leq \lambda ( \bigcup_{l=1}^k A_{j_l} ) \leq \sum_{l=1}^k \lambda(A_{j_l})$ par \textbf{thm. 8.6}.
\\
\\
Si on permet les fonctions négatives, je crois que le théorème est faux. On pose $s_1 := -\chi_{\mathbb{R}}$ et $s_2 := \chi_\mathbb{R}$. Alors $s_1 + s_2 = 0$. On a donc $\int_\mathbb{R} (s_1 + s_2)d\lambda = 0$. Or $\int_\mathbb{R} s_1 d\lambda + \int_\mathbb{R} s_2 d\lambda = -\infty + \infty$, qui est une forme indeterminée.
\section*{9-21}
\begin{enumerate}
	\item $f : \mathbb{R} \rightarrow [0, \infty)$ bornée et Lebesgue mesurable
	\item $\lambda \left( \{ x \in \mathbb{R} : f(x) > 0 \} \right) < \infty$
\end{enumerate}
$\diamond$ \textbf{$f$ est Lebesgue intégrable et }
\begin{align*}
	\int_\mathbb{R} f d \lambda = \inf \left \{ \int_\mathbb{R} s d \lambda : s \text{ est simple et } f \leq s \right \}
\end{align*}
Puisque $f$ est bornée, supposons $M$ tel que $f(x) \leq M$ pour tout $x$.
\\
\\
On a de plus que $f(x) \leq M \chi_{\{f > 0\}}$ pour tout $x$. Or, le membre de droit est une fonction simples et son intégral est donnée par $M \lambda ( \{f > 0 \} ) < \infty$.
\\
\\
Par \textbf{thm. 9.23}, on a $\int_\mathbb{R} f d\lambda \leq M \lambda ( \{f > 0 \} )$ et donc $f$ Lebesgue intégrable.
\\
\\
Il est clair que $\int_\mathbb{R} f d\lambda \leq \inf \{ \cdots \}$. On prouve donc qu'il ne peut pas être strictement plus grand.
\\
\\
On définit
\begin{align*}
	h(x) := M \chi_{\{f > 0\}} - f
\end{align*}
Alors $h \leq M$ et $\{h > 0\} = \{f > 0\}$. De plus, $h \leq M\chi_{\{f > 0\}}$ et donc, par \textbf{thm. 9.23}, $h$ est Lebesgue intégrable.
\\
\\
Par \textbf{def. 9.22}, on a que pour tout $\epsilon > 0$ il existe $s \leq h$ tel que $\int_\mathbb{R} h -s d\lambda < \epsilon$. Or
\begin{align*}
	& h - s = M\chi_{\{f > 0\}} - f - s \\
	= \\
	& (M\chi_{\{f > 0\}} - s) - f \\
	\Rightarrow \\
	& \int_\mathbb{R} (M \chi_{\{f > 0\}} - s)d\lambda - \int_\mathbb{R} f d\lambda < \epsilon
\end{align*}
Or $M\chi_{\{f > 0 \}} - s$ est une fonction escalier tel que $M\chi_{\{f > 0\}} - s \geq f$ puisque $s \leq M\chi_{\{f > 0\}} - f \Leftrightarrow M\chi_{\{f > 0\}} - s \geq f$.
\\
\\
Donc, l'infimum considéré ne peut pas être plus grand que $\int_\mathbb{R} f d\lambda$, donc il est égal.
\section*{9-22}
\begin{enumerate}
	\item $f : \mathbb{R} \rightarrow [0, \infty]$ Lebesgue mesurable
\end{enumerate}
$\diamond$ \textbf{$f$ est Lebesgue intégrable ssi}
\begin{align*}
	S := \sup \left \{ \int_\mathbb{R} \min (f,n) \chi_{[-n,n]}d\lambda : n \in \mathbb{N} \right\}
\end{align*}
\textbf{est fini et alors $S$ est l'intégrale de Lebesgue de $f$}
\\
\\
On suppose $f$ Lebesgue intégrable.
\\
\\
Alors pour tout $n$ on a $0 \leq \min (f,n) \chi_{[-n,n]} \leq f$. Par le \textbf{thm. 9.23}, $\min (f,n) \chi_{[-n,n]}$ est Lebesgue intégrable. 
\\
\\
Puisque $f$ est Lébesgue intégrable, on a que pour tout $\epsilon > 0$ il existe un $0 \leq s \leq f$ une f.e. tq $\int_{\mathbb{R}} f - s d \lambda < \epsilon$. (\textbf{def. 9.22}) 
\\
\\
De plus, par \textbf{def. 9.15}, on a qu'il existe un $n$ tel que $s \leq n$. 
\\
\\
On a donc que $\min (f,n) \chi_{[-n,n]} - s \leq f - s$ et donc $\int_\mathbb{R} \min (f,n) \chi_{[-n,n]} - s d\lambda \leq \int_\mathbb{R} f - s d\lambda$. 
\\
\\
On a 
\begin{align*}
	& |\min (f,n) \chi _{[-n,n]} - s| \\
	\leq \\
	& |f-s|\chi_{[-n,n] \cap \{f \leq n\}} + |n - s|\chi_{[-n,n] \cap \{f > n\}} \\
	= \\
	& (f-s)\chi_{[-n,n] \cap \{f \leq n\}} + (n - s)\chi_{[-n,n] \cap \{f > n\}} \\
	\leq \\
	& (f-s)\chi_{[-n,n] \cap \{f \leq n\}} + (f - s)\chi_{[-n,n] \cap \{f > n\}} \\
	= \\
	& (f-s) \chi_{[-n,n]} \\
	\leq \\
	& f-s = |f-s|
\end{align*}
On déduit 
\begin{align*}
	& 2 \epsilon \geq \int_{\mathbb{R}} |f-s| + |s - \min (f,n) \chi_{[-n,n]}| d\lambda \\
	\geq \\
	& \int_\mathbb{R} |f - \min (f,n) \chi_{[-n,n]}| d\lambda \\
	= \\
	& \int_\mathbb{R} f - \min (f,n) \chi_{[-n,n]} d\lambda
\end{align*}
Donc, pour tout $\epsilon$, il existe un $n$ tel que $\int_\mathbb{R} f - \min (f,n) \chi_{[-n,n]} d\lambda < \epsilon$. Donc $\int_\mathbb{R} \min (f,n) \chi_{[-n,n]} d\lambda$ tend vers $\int_\mathbb{R} f d\lambda$ et donc $\int_\mathbb{R} f d\lambda = S$ (\textbf{s.d.}).
\\
\\
Supposons alors $S < \int_\mathbb{R} f d\lambda$.
\\
\\
Alors il existe $s$ tel que $S < \int_\mathbb{R} s d\lambda \leq \int_\mathbb{R} f d\lambda$ (\textbf{def. 9.22}).
\\
\\
Or, il existe $m$ tel que $s \chi_{[-m,m]} \leq \min (f,m) \chi_{[-m,m]}$ (\textbf{def. 9.15}). Donc $\int_\mathbb{R} s \chi_{[-n,n]} d\lambda \leq S < \int_\mathbb{R} f d\lambda$ et ce pour tout $n \geq m$.
\\
\\
Or, je dis qu'il existe $n > m$ tel que $S < \int_\mathbb{R} s \chi_{[-n,n]} d\lambda \leq \int_\mathbb{R} f d\lambda$. 
\\
\\
Car $\lim_{n \to \infty} \int_\mathbb{R} s \chi_{[-n,n]} d\lambda = \lim_{n \to \infty} \sum_{i=1}^k a_i \lambda (A_i \cap [-n,n]) = \sum_{i=1}^k a_i \lim_{n\to \infty} \lambda( A_i \cap [-n,n])$ (\textbf{thm. 2.14}).
\\
\\
Or $A_i \cap [-n,n] \subseteq A_i \cap [-n-1, n+1]$ pour tout $n$. Donc, par \textbf{thm. 9.12}, on a $\lim_{n \to \infty} \lambda (A_i \cap [-n,n]) = \lambda \left(\bigcup^\infty_{n=1} A_i \cap [-n,n] \right) = \lambda \left( A_i \cap \left(\bigcup_{n=1}^\infty [-n,n] \right) \right) = \lambda (A_i \cap \mathbb{R}) = \lambda (A_i)$.
\\
\\
Ainsi il existe un $n > m$ tel que $S < \int_\mathbb{R} s\chi_{[-n,n]} d\lambda \leq \int_\mathbb{R} f d\lambda$, ce qui est impossible.
\\
\\
Donc $\int_\mathbb{R} f d\lambda \leq S$. Or, puisque $\min (f,n) \chi_{[-n,n]} \leq f$ pour tout $f$, on $\int_\mathbb{R} \min (f,n) \chi_{[-n,n]} d\lambda \leq \int_\mathbb{R} f d\lambda$. En prenant le sup des deux côté, on a que $S \leq \int_\mathbb{R} f d\lambda$. Donc $S = \int_\mathbb{R} f d\lambda$.

\section*{9-23}
\begin{enumerate}
	\item $f: \mathbb{R} \rightarrow [0, \infty]$ Lebesgue intégrable
	\item $a > 0$
	\item $A \subseteq \mathbb{R}$ Lebesgue mesurable
	\item $f - \chi_A \leq 0$
\end{enumerate}
$\diamond$ \textbf{$\int_\mathbb{R} f - a\chi_A d\lambda = \int_\mathbb{R} f - a \lambda (A)$}
\\
\\
On a que $a\chi_A$ est une fonction simple et son intégrale est donnée pas $a \lambda (A)$ (\textbf{def. 9.21}). Donc, par \textbf{thm. 9.25}, on peut conclure.

\section*{9-24}
$\diamond$ \textbf{Construire deux fonctions $f,g :\mathbb{R} \rightarrow \mathbb{R}$ tel que $(f+g)^+ \neq f^+ + g^+$ et $(f + g)^- \neq f^- + g^-$}
\\
\\
$f := 1$ et $g := -1$.

\section*{9-25}
\begin{enumerate}
	\item $a_1 \cdots a_n \in \mathbb{R}$
	\item $A_1 \cdots A_n \subseteq \mathbb{R}$ pas nécessairement disjoints des ensembles Lebesgue mesurables
	\item $f := \sum_{k=1}^n a_k \chi_{A_k}$
\end{enumerate}
$\diamond$ \textbf{$\int_\mathbb{R} f d\lambda  = \sum_{k=1}^n a_k \lambda (A_k)$. Expliquez en quoi ce résultat diffère de ce qui fut démontré en 9-20a et pour quelles raisons il aurait été impossible de s'en servir pour prouver 9-20b.}
\\
\\
Par \textbf{thm. 9.25}, on a immédiatement le résultat voulu.
\\
\\
TODO

\section*{9-26}
\begin{enumerate}
	\item $f,g : \mathbb{R} \rightarrow [-\infty, \infty]$ Lebesgue mesurables
	\item $f = g$ pp
\end{enumerate}
$\diamond$ \textbf{En se servant seulement du thm. 9.23, montrez que $f$ est Lebesgue intégrables ssi $g$ est Lebesgue intégrable et alors $\int_\mathbb{R} f d\lambda = \int_\mathbb{R} g d\lambda$}
\\
\\
Supposons $f$ Lebesgue intégrable. On a que $f^+ = g^+$ pp et $f^- = g^-$ pp.
\\
\\
Or, par \textbf{thm. 9.23}, on a que $|f|$ est Lebesgue intégrable et $f^+ \leq |f|$. Par le même théorème, $f^+$ est intégrable. On résone analoguement pour $f^-$.
\\
\\
On a donc, par conséquence du \textbf{thm. 9.23}, que $\int_\mathbb{R} f^+ d\lambda = \int_\mathbb{R} g^+ d\lambda$ et $\int_\mathbb{R} f^- d\lambda$. Ceci montre que $g$ est Lebesgue intégrable par \textbf{def. 9.22}. Par la même définition, on déduit $\int_\mathbb{R} f d\lambda = \int_\mathbb{R} f^+ d\lambda - \int_\mathbb{R} f^- d\lambda = \int_\mathbb{R} g^+ d\lambda - \int_\mathbb{R} g^- d\lambda = \int_\mathbb{R} g d\lambda$.

\section*{9-27}
\begin{enumerate}
	\item $f : \mathbb{R} \rightarrow [-\infty, \infty]$ Lebesgue intégrable
\end{enumerate}
$\diamond$ \textbf{$\{x \in \mathbb{R} : f(x) = \infty\}$ est de mesure nulle}
\\
\\
Car supposons le contraire. Alors $|f|$ est Lebesgue intégrable par \textbf{thm. 9.23} et de plus $|f|\chi_{\{f = \infty\}} \leq |f|$. Donc le membre de gauche de cette inéquation est intégrable par \textbf{thm. 9.23} et $\int_\mathbb{R} |f|\chi_{\{f = \infty\}} d\lambda \leq \int_\mathbb{R} |f| d\lambda$. Or $\int_\mathbb{R} |f| \chi_{\{f = \infty\}} d\lambda = \infty$.
\\
\\
En effet, la fonction $s_n := n \chi_{\{f = \infty\}}$ est une suite de fonctions simples tel que $0 \leq s_n \leq |f|\chi_{\{f = \infty\}}$ pour tout $n$. Or $\int_\mathbb{R} n \chi_{\{f = \infty\}} d\lambda = n \lambda(\{f = \infty\}) \rightarrow \infty$. 

\section*{9-28}
\begin{enumerate}
	\item $f,g : \mathbb{R} \rightarrow [-\infty, \infty]$ des fonctions Lebesgues intégrables
\end{enumerate}
\subsection*{a)}
$\diamond$ \textbf{$\max (f,g)$ est Lebesgue intégrable}
\\
\\
On a que $\max (f,g) = f \chi_{\{f \geq g\}} + g \chi_{\{f < g\}}$. On a que $0 \leq f^+ \chi_{\{f \geq g\}} \leq f^+$ et $0 \leq f^- \chi_{\{f \geq g\}} \leq f^-$ et donc par \textbf{thm. 9.23} on a deux fonctions intégrables. Donc, par \textbf{def. 9.22}, $f \chi_{\{f \geq g\}}$ est intégrable. De même pour $g\chi_{\{f < g\}}$.
\\
\\
Par \textbf{thm. 9.25}, la fonction $\max (f,g)$ est intégrable.  

\subsection{b)}
$\diamond$ \textbf{$\min (f,g)$ est Lebesgue intégrable}
\\
\\
Même chose qu'en a).

\subsection*{c)}
$\diamond$ \textbf{$f-g$ est Lebesgue intégrable et $\int_\mathbb{R} f-g d\lambda = \int_\mathbb{R} f d\lambda - \int_\mathbb{R} g d\lambda$}
\\
\\
Par \textbf{thm. 9.25}, $-g$ est intégrable et donc $f + (-g)$ l'est également. De plus, on a $\int_\mathbb{R} f + (-g) d\lambda = \int_\mathbb{R} f d\lambda + \int_\mathbb{R} (-g)d\lambda$. On ré-applique \textbf{thm. 9.25} pour avoir le résultat voulu.

\section*{9-29}
\begin{enumerate}
	\item $C^Q$ l'ensemble de Cantor. 
\end{enumerate}
$\diamond$ \textbf{$\chi_{C^Q}$ est Lebesgue intégrable}
\\
\\
On a qu'il s'agit d'une fonction simple et, de plus, par \textbf{ex. 9-6}, l'ensemble de Cantor est Lebesgue mesurable. Donc l'intégrale est égal à $\lambda (C^Q)$ par \textbf{def. 9.21-22}. Or, le Camtor est définie sur un interval fermé $[a,b]$ (\textbf{def. 7.22}). Donc la fonction est Lebesgue intégrable et l'intégrale est finie (\textbf{thm. 9.23}). 

\section*{9-30}
$\diamond$ \textbf{$\chi_{\mathbb{Q} \cap [0,1]}$ est Lebesgue intégrable}
\\
\\
Premièrement, $\mathbb{Q} \cap [0,1]$ est mesurable car $[0,1]$ est mesurable par \textbf{prop. 9.13} et $\mathbb{Q}$ est mesurable car dénombrable et donc de mesure nulle (\textbf{prop. 8.3, prop. 9.7}). Donc l'intersection est Lebesgue mesurable par \textbf{lem. 9.8}.
\\
\\
On a que $\chi_{\mathbb{Q} \cap [0,1]} \leq \chi_{[0,1]}$ qui est Lebesgue intégrable. On applique \textbf{thm. 9.23}.
\end{document}
