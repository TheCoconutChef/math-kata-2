\documentclass[a4paper,10pt]{article}
\usepackage[utf8]{inputenc}
\usepackage[T1]{fontenc} % Use 8-bit encoding that has 256 glyphs
\usepackage{ae, aecompl}
\usepackage{amsmath,amsfonts,amsthm} % Math packages
\usepackage{enumitem}
\usepackage[parfill]{parskip}
\usepackage{fancyhdr} % Custom headers and footers
\setlength{\parindent}{0cm}

%opening
\title{3.3 Wilson's theorem}

\begin{document}

\maketitle

\section*{1}
$\diamond$ $p$ \textbf{est le plus petit nombre premier divisant} $(p-1)! + 1$. 
\\
\\
On a que si $a$ divise $b$ et $b-1$, alors $a = 1$, car alors $a$ divise $b - (b-1) = 1$.
\\
\\
Soit alors $p^* < p$ un premier divisant $(p-1)! + 1$. Puisque $p^*$ divise $(p^* - 1)! + 1$ par Wilson, on 
a que $p^*$ divise également $(p-1)! + 1 - ((p^* - 1)! + 1) = (p-1)! - (p^* - 1)! = (p^* - 1)!((p-1) \cdots (p^*) - 1)$.
\\
\\
Or, $p^*$ ne peut pas diviser $(p^* - 1)!$ puisqu'il divise $(p^* - 1)! + 1$, et s'il le fesait il serait alors égal à 1.
\\
\\
Il doit donc diviser $((p-1) \cdots (p^*) - 1)$ (\textbf{justification}). Mais $p^*$ divise $(p-1) \cdots (p^*)$. Alors divisant
$(p-1) \cdots (p^*) -1$, il doit être égal à 1, ce qui est absurde.
\\
\\
Alors $p$ est le plus petit premier divisant $(p-1)! + 1$.

\section*{2}
$\diamond$ 10 $\not | [(n-1)! + 1]$ pour tout $n$.
\\
\\
Car SLC. Alors $10k = (n-1)! + 1$ pour un certain $n > 4$. Alors $(n-1)! + 1$ est pair, car divisble par 2.
\\
\\
Or, $(n-1)!$ est pair pour tout $n > 1$. Donc $(n-1)! + 1$ est impair, et donc contradiction. 
\end{document}
