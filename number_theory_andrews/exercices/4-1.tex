\documentclass[a4paper,10pt]{article}
\usepackage[utf8]{inputenc}
\usepackage[T1]{fontenc} % Use 8-bit encoding that has 256 glyphs
\usepackage{ae, aecompl}
\usepackage{amsmath,amsfonts,amsthm} % Math packages
\usepackage{enumitem}
\usepackage[parfill]{parskip}
\usepackage{fancyhdr} % Custom headers and footers
\setlength{\parindent}{0cm}

%opening
\title{4.1 Basic properties of congruence}

\begin{document}

\maketitle

\section*{2}
\subsection*{a)}
$\diamond$ \textbf{Existe-t-il un $x$ tel que $6x \equiv 5(\text{mod } 4)$?}
\\
\\
Premièrement, on a qu'un nombre pair ne peut pas diviser un nombre impair. Car soit $\frac{2k+1}{2n} = d$. Alors $\frac{k}{n} + \frac{1}{2n} = d$ et donc $k + \frac{1}{2} = dn$, ce qui est impossible.
\\
\\
On a ensuite que $6x - 5 = 6(x-1) + 1$ pour $x > 1$. Or, $6(x-1) + 1$ est impair et 4 est pair. Donc 4 ne divise pas $6(x-1)+ 1$ pour $x > 1$. Si $x = 1$, alors $6-5 = 1$ qui n'est pas divisible par 4.

\subsection*{b)}
$\diamond$ \textbf{Existe-t-il un $x$ tel que $10x \equiv 8 (\text{mod }6)$?}
\\
\\
Avec $x = 2$, on a $20 - 8 = 12$ divisible par 6.

\subsection*{c)}
$\diamond$ \textbf{Existe-t-il un $x$ tel que $12x \equiv 9 (\text{mod }6)$?}
\\
\\
Un raisonnement analogue à celui employé en $a)$ peut-être appliqué. On a que $12x - 9 = 12(x-1) + 3$ un nombre impair. Or 6 est pair et donc ne peut pas le diviser. Si $x = 1$, alors on a que $12-9 = 3$ n'est pas divisible par 6.

\section*{3}
\begin{enumerate}
	\item $x \equiv y (\text{mod } m)$
	\item $a_0, a_1, \cdots, a_r \in \mathbb{Z}$
\end{enumerate}
$\diamond$ \textbf{$a_0x^r + a_1 x^{r-1} + \cdots + a_r \equiv a_0 y^r + \cdots a_r (\text{mod }m)$}
\\
\\
On a que $a_i \equiv a_i (\text{mod } m)$ par \textbf{thm 4.1}. De plus, par \textbf{thm. 4.2}, on a que $x^i \equiv y^i (\text{mod } m)$ pour tout $i \in \mathbb{N}$.
\\
\\
Donc, par $\textbf{thm. 4.2}$, on a que $a_i x^{r-i} \equiv a_i y^{r-i}(\text{mod } m )$ et de même pour la somme $a_0 x^r + \cdots a_r \equiv a_0 y^r + \cdots a_r (\text{mod }m)$

\section*{4}
\begin{enumerate}
	\item $bd \equiv bd' (\text{mod } p)$
	\item $p$ premier et ne divise pas $b$
\end{enumerate}
$\diamond$ \textbf{$d \equiv d' (\text{mod }p)$}
\\
\\
Puisque $p$ est premier, on a que $(b,p) = 1$. On applique alors le \textbf{thm. 4.3}.

\section*{5}
\begin{enumerate}
	\item $|a|, |b| < \frac{k}{2}$
	\item $a \equiv b(\text{mod } k)$
\end{enumerate}
$\diamond$ \textbf{$a = b$}
\\
\\
On a que $\frac{a-b}{k} \leq \frac{|a-b|}{k} \leq \frac{|a|}{k} + \frac{|b|}{k} < \frac{1}{2} + \frac{1}{2} = 1$. Donc $\frac{a-b}{k} = 0$.
\end{document}
