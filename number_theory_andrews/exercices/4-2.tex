\documentclass[a4paper,10pt]{article}
\usepackage[utf8]{inputenc}
\usepackage[T1]{fontenc} % Use 8-bit encoding that has 256 glyphs
\usepackage{ae, aecompl}
\usepackage{amsmath,amsfonts,amsthm} % Math packages
\usepackage{enumitem}
\usepackage[parfill]{parskip}
\usepackage{fancyhdr} % Custom headers and footers
\setlength{\parindent}{0cm}

%opening
\title{4.2 Residue system}

\begin{document}

\maketitle

\section*{3}
\subsection*{a)}
\begin{enumerate}
	\item $A := \{a_1, \cdots, a_k\}$ un système de résidus complet
	\item $k$ premier
\end{enumerate}
$\diamond$ \textbf{Pour tout $n,s$ il existe $b_1, \cdots b_j \in A$ tel que $n \equiv \displaystyle\sum_{j=0}^s b_j k^j (\text{ mod }k^{s+1})$}
\\
\\
La chose est vraie pour $s = 0$ par définition. Soit la chose prouvé pour $s$. 
\\
\\
Alors on a que $n - \displaystyle\sum_{j=0}^s b_j k^j = d_s k^{s+1}$ pour un certain $d_s$.
\\
\\
Mais on a que $d_s = d_{s+1}k + b_{s+1}$ (puisque l'on a un système de résidue complet) où $b_{s+1} \in A$. Donc $d_s k^{s+1} = d_{s+1}k^{s+2} + b_{s+1}k^{s+1} \Leftrightarrow d_{s}k^{s+1} - b_{s+1}k^{s+1} = d_{s+1}k^{s+2}$. On substitut $d_{s}k^{s+1}$ et on obtient $n - \displaystyle\sum_{j=0}^s b_j k^j - b_{s+1}k^{s+1} = n - \displaystyle\sum_{j=0}^{s+1} b_j k^j = d_{s+1}k^{s+2}$.

\section*{4}
\begin{enumerate}
	\item $w(n) := |\{p : p \text{ premier et } p \not | n\}|$
\end{enumerate}
$\diamond$ \textbf{A-t-on $w(n) < \phi(n)$? Existe-t-il un $n$ tel que $w(n) = \phi(n)-1$}
\\
\\
Premièrement, on a que tout $p$ ne divisant pas $n$ sera co-premier à $n$ et donc $w(n) \leq \phi(n)$. Par contre, 1 est co-premier à tout $n$ sans être premier et donc $w(n) < \phi(n)$.
\\
\\
De plus, si $n=3$, on a que $w(n) = 1$ et $\phi(n) = 2$.
\end{document}
