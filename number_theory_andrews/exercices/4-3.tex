\documentclass[a4paper,10pt]{article}
\usepackage[utf8]{inputenc}
\usepackage[T1]{fontenc} % Use 8-bit encoding that has 256 glyphs
\usepackage{ae, aecompl}
\usepackage{amsmath,amsfonts,amsthm, amssymb} % Math packages
\usepackage{enumitem}
\usepackage[parfill]{parskip}
\usepackage{fancyhdr} % Custom headers and footers
\setlength{\parindent}{0cm}

%opening
\title{4.3 The theorem of Fermat and Wilson revisited}

\begin{document}

\maketitle

\section*{3}
\subsection*{a)}
\begin{enumerate}
	\item $(a,m) = 1$
	\item $(a-1,m) = 1$
\end{enumerate}
$\diamond$ \textbf{$1 + a + a^2 + \cdots + a^{\phi(m) - 1} \equiv 0 (\textbf{mod } m)$}
\\
\\
Car on a que $\displaystyle\sum_{j=0}^{\phi(m)-1} a^j = \frac{1-a^{\phi(m)}}{1-a}$.
\\
\\
Donc $(a-1) \displaystyle\sum_{j=0}^{\phi(m)-1} a^j = a^{\phi(m)} - 1$. Or, par le théorème d'Euler, on a que $a^{\phi(m)}-1 \equiv 0 (\text{mod } m)$. Donc $(a-1)\displaystyle\sum_{j=0}^{\phi(m)-1} a^j \equiv 0 (\text{mod } m)$. Puisque $(a-1,m) = 1$, on peut retirer le terme $(a-1)$ et alors on a le résultat voulu.

\section*{4*}
\begin{enumerate}
	\item $r_1, \cdots, r_{\phi(m)}$ un système de résidue réduit
	\item $m$ impair
\end{enumerate}
$\diamond$ \textbf{$r_1 + r_2 + \cdots + r_{\phi(m)} \equiv 0 (\text{mod } m)$}
\\
\\
Puisque $m$ est impair, on a que $\text{pgdc}(m,2) = 1$ et donc $2r_i$ possède un résidue unique dans le système puisqu'il s'agit d'un nombre relativement premier à $m$. Il en va de même pour chaque $2r_j$ auquel correspond un unique résidue.
\\
\\
Ansi, on a que $2\displaystyle\sum_{j=1}^{\phi(m)}r_j \equiv \displaystyle\sum_{j=1}^{\phi(m)} r_j (\text{mod }m)$ et donc $\displaystyle\sum_{j=1}^{\phi(m)}r_j \equiv 0 (\text{mod }m)$.  

\section*{11}
\begin{enumerate}
	\item $p$ premier et différent de 2
	\item $0 \leq a \leq p-1$
\end{enumerate}
$\diamond$ \textbf{$\binom{p-1}{a} \equiv (-1)^a (\text{mod } p)$}
\\
\\
La chose est évidente pour $\binom{p-1}{0}$ et $\binom{p-1}{p-1}$. Supposons la chose vraie pour $a-1$ et considérons $0 < a < p-1$.
\\
\\
On a 
\begin{align*}
	& \binom{p-1}{a} - (-1)^a = \binom{p-1}{a-1}\frac{p-1}{a} - (-1)^a \\
	= \\
	& \frac{\binom{p-1}{a-1}p - \binom{p-1}{a-1}a - (-1)^a a}{a} \\
	= \\
	& \binom{p-1}{a-1}\frac{p}{a} - \left[\binom{p-1}{a-1} + (-1)^{a} \right] \\
	= \\
	& \binom{p}{a} + \left[\binom{p-1}{a-1} - (-1)^{a-1} \right]
\end{align*}
Or, on a que $\binom{p}{a}$ est divisible par $p$ (\textbf{ex. 3-1 9 et $0 < a < p-1$}) et $\binom{p-1}{a-1} - (-1)^{a-1}$ l'est également par hypothèse d'induction. 

\section*{12}
\begin{enumerate}
	\item $n < p \leq 2n$
	\item $p$ premier
\end{enumerate}
$\diamond$ \textbf{$p$ divise $\binom{2n}{n}$ mais pas $p^2$}
\\
\\
On a que $\binom{2n}{n} = p \frac{2n \cdots (p+1)(p-1) \cdots (n+1)}{n!} = d$ et donc $2n \cdots (p+1)(p-1) \cdots (n+1) = \frac{n!d}{p}$, d'où l'on conclut que $\frac{n!d}{p}$ est un entier et donc $p$ divise $n!d$.
\\
\\
Or $p \nmid n, (n-1), \cdots 2$ puisque $p$ est premier. Donc $p$ divise $d$ par \textbf{cor. 2-3}. Alors $\frac{2n \cdots (p+1)(p-1) \cdots (n+1)}{n!} = \frac{d}{p}$ un entier. Mais alors $\binom{2n}{n} = p \frac{2n \cdots (p+1)(p-1) \cdots (n+1)}{n!} = pk$ où $k$ est un entier. Donc $p \mid \binom{2n}{n}$.
\\
\\
Supposons alors que $p^2$ divise $\binom{2n}{n}$. On a que $kp^2 = \frac{2n!}{(n!)^2} \Leftrightarrow lp^2 = 2n!$ pour un certain $k$. Donc $p^2 \mid 2n!$.
\\
\\
Alors $lp = 2n \cdots (p+1)(p-1)! = 2n \cdots (p+1)(p\alpha -1)$ pour un certain entier $\alpha$ par \textbf{thm. 5-3}. Or $p \nmid (p\alpha - 1)$. De plus $n < p < 2n < 2p$. Or, par \textbf{cor. 2-3 et cor. 2-4}, $p$ doit diviser $p+1$ ou $p+2$ ou ... $2n$, ce qui est impossible.

\section*{14}
$\diamond$ \textbf{Calculer la quantité de nombre premier comprise dans l'interval fermé $[m! + 2, m! + m]$ où $m > 1$}
\\
\\
Soit $0 \leq k \leq m-2$. Alors on a que $m! + 2 \leq m! + 2 + k \leq m! + m$. Or, on a que $m! + 2 + k = m! + q = m \cdots q! + q$ et donc pour tout $q = 2 +k$, on a que $m! + 2 + k$ est divisible par $q$. Or, $q < m! + q$ et donc $m! + q = dq$ où $d > 1$. Donc $m! + q$ n'est pas premier et ce pour tout $k$ entre $0$ et $m -2$.
\end{document}
